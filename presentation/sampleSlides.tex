\documentclass{beamer} % comment this line out before building notes file
\usepackage{amsmath,amsthm,amssymb}   % make math look better
% \usepackage{algorithm, algpseudocode} % allows pseudo-code blocks
\usepackage[caption=false]{subfig} % allows side-by-side figures
\usepackage[font=scriptsize,labelfont=bf]{caption} % fixes captions with beamer's weirdness
% \usepackage{listings} % for listing code blocks
% \lstset{basicstyle=\ttfamily\footnotesize,breaklines=true}
\usepackage{booktabs} % book tables are simple and look nice
\usepackage{xcolor}   % for template colors
\usepackage{graphicx} % for images, etc.

% Change template colors to CSU colors, notation is {color!%opacity}
\definecolor{CSUgreen}{RGB}{30,77,43}   % define green
\definecolor{CSUtan}{RGB}{215,211,146}  % define tan
\setbeamercolor{structure}{fg=CSUgreen} % 'structure' handles most of the colors
% \setbeamercolor{background canvas}{bg=CSUtan!18} % set background color
\setbeamercolor{palette sidebar secondary}{fg=CSUgreen} % highlighted section
\setbeamercolor{section in sidebar shaded}{fg=CSUgreen!70} % faded sections
\setbeamercolor{block title}{bg=CSUgreen!40,fg=CSUgreen}
\setbeamercolor{block body}{bg=CSUgreen!20}
\setbeamercolor{block title example}{bg=CSUtan!70,fg=CSUgreen}
\setbeamercolor{block body example}{bg=CSUtan!50}
\usetheme{Goettingen}
\makeatletter % allows you to edit macros of theme
\setbeamertemplate{sidebar canvas \beamer@sidebarside}[vertical shading][top=CSUgreen!50,bottom=CSUgreen!30]
\makeatother

\setbeamertemplate{caption}{\insertcaption}

\newcommand*\dif{\mathop{}\!\mathrm{d}}         % get a straight d for differential in integrals
\newcommand{\abs}[1]{\left\lvert#1\right\rvert} % absolute value
\makeatletter                                   % Hack the \vdots command to take up less space
\DeclareRobustCommand{\svdots}{%
  \vbox{
    \baselineskip4\p@\lineskiplimit\z@
    \kern-\p@
    \hbox{.}\hbox{.}\hbox{.}
  }}

% \setcounter{MaxMatrixCols}{20} % allow more table entries to display
\makeatother
\graphicspath{{../../figures/}} % Set path for graphics

%-------------------------------------------------------------------------------
%	TITLE PAGE
%-------------------------------------------------------------------------------

\title[Short Title]{Full Title}
\author[Short Name]{Full Name \\ \vspace{.2cm}
{\scriptsize Colorado State University\\
email@colostate.edu}
}

\institute[CSU]
{\begin{tabular}{r@{}l}
Advisor:   \  & Advisor \\
Committee: \  & Committee Member 1 \\
              & Committee Member 2 \\
              & Committee Member 3
\end{tabular}
}
\date{Date}

\begin{document}

\begin{frame}
% \nocite{*} % Prints all references in bibliography
\titlepage % Print the title page as the first slide
\end{frame}
%=============================================================================
% These notes will build as a separate file. Every slide needs one, even if it's blank.
\note[itemize]{
\item Thank everyone for coming. Thank committee.

\item Good luck! :)
}

%-------------------------------------------------------------------------------
%	PRESENTATION SLIDES
%-------------------------------------------------------------------------------
\section{First Section}
\subsection{First Subsection}

\begin{frame}
  \frametitle{Generic Slide}
    \begin{figure}
    % \includegraphics[scale=.35]{cartoon.jpg}
    \caption{Caption.}
  \end{figure}
\end{frame}
%=============================================================================
\note{ Notes. }

\section{References}

\begin{frame}[allowframebreaks] % references flow onto multiple slides
\frametitle{References}

\bibliographystyle{siam}
% {\footnotesize \bibliography{../../Bibliography.bib}}
\end{frame}
\note{The End.}

\end{document}
